\documentclass[10pt,leqno,twoside]{article}

\usepackage{annal-latex}
\usepackage{amsmath}
\usepackage{booktabs}
\usepackage{graphicx}
\usepackage{tikz}
\usepackage{draftwatermark}
\usepackage{color}
\usepackage[backend=bibtex, sortcites=true, style=numeric-comp, sorting=none]{biblatex}
\usepackage{authblk}



\addbibresource{research.bib}

\SetWatermarkText{\textbf{FAKE RESEARCH}}
\SetWatermarkLightness{0.85}
\SetWatermarkAngle{60}
\SetWatermarkScale{3.0}

\begin{document}
\setcounter{page}{1}

\vspace{-4cm} \fofej{49}{19}{nn}{nnn}

\vspace{.4cm}


\title{Achieving Near-Zero Hallucation In Large Language Models}

\author{{\bf Name} (City, Country)\\[1ex]
{\bf Name} (City, Country)
{\bf Name} (City, Country)
{\bf Name} (City, Country)}

\date{}



\abstract{
This paper presents a research methodology focused on the mitigation of hallucinations in modern large language models.  The initial phase involved the development and training of a models following the framework established in Google’s “Attention Is All You Need” \cite{vaswani2017attention} paper. The degree of hallucination present in the model outputs was then systematically measured with respect to the training data. These measurements were obtained after multiple training iterations conducted on models of identical size but trained on datasets differing in quality and factual reliability, thereby yielding models with varying levels of knowledge representation. The results indicated that both data quality and model architecture contributed significantly to the prevalence of hallucinations. Consequently, architectural modifications were introduced, after which a model was trained on high-quality data. This revised configuration achieved a reduction of hallucinations in 96.4\% of test cases.
\textcolor{red}{\newline This research paper is for a Research Methodology course at ELTE University. The data in it is made up, and should not be taken seriously or referenced.}}


% ---------------------------------------------------------------


\section{Introduction}
Your introduction

\subsection{Related work}
Works related to your paper

\section{Methodology}
\label{sec:methodology}

The experimental investigation was conducted in two sequential phases: first, establishing a robust performance baseline and quantifying the relationship between training data reliability and hallucination prevalence \cite{islam2024comprehensive, cossio2025comprehensive, cao2021hallucination}; and second, implementing and evaluating the novel architectural intervention.

\subsection{Phase 1: Baseline Establishment and Data Quality Analysis}
\label{sec:phase1}

\subsubsection{Model Initialization and Data Corpus Formalization}
\label{sec:model_data_formalization}
All comparative experiments utilized the \textbf{Canonical Transformer Model} as the foundational architecture, strictly instantiated according to the specifications of \cite{vaswani2017attention}. A standard configuration was employed (e.g., \textbf{6 encoder layers, 6 decoder layers, and 8 attention heads}) \cite{vaswani2017attention}. All initializations used a standard parameter initialization and an Adam optimizer with the inverse square-root learning rate schedule \cite{vaswani2017attention}.

The experimental corpora were constructed from large-scale, publicly available datasets, segregated into two categories based on empirical reliability ratings:

\begin{itemize}
    \item \textbf{Low-Reliability Corpus} ($\mathcal{D}_{\text{LR}}$): Comprised of tokens sampled from \textit{Reddit (2020 snapshot)}, \textit{Common Crawl WET files}, and \textit{WikiAnswers community data}. Prior work (\cite{cao2021hallucination, islam2024comprehensive, cossio2025comprehensive}) has demonstrated high factual heterogeneity and weak source attribution in these domains.
    \item \textbf{High-Reliability Corpus} ($\mathcal{D}_{\text{HR}}$): Comprised of tokens from \textit{English Wikipedia (2023-06 dump)}, the \textit{Stanford Question Answering Dataset (SQuAD v2)} \cite{rajpurkar2018squadv2}, and the \textit{Natural Questions Open dataset} \cite{kwiatkowski2019natural}. These sources underwent strict deduplication and factuality validation, exhibiting high human-rated factual consistency in preliminary audits.
\end{itemize}

\subsubsection{Differential Baseline Training and Factual Integrity Quantification}
\label{sec:baseline_training_quantification}
Two baseline models were trained to empirically isolate the causal effect of training data reliability on factual consistency. Both followed an \textbf{identical training regimen} (e.g., same number of training steps, batch size, and dropout rate).

\begin{itemize}
    \item \textbf{Baseline $B_1$}: Trained exclusively on the Low-Reliability Corpus ($\mathcal{D}_{\text{LR}}$).
    \item \textbf{Baseline $B_2$}: Trained exclusively on the High-Reliability Corpus ($\mathcal{D}_{\text{HR}}$).
\end{itemize}

Factual adherence was quantified via the \textbf{Hallucination Rate} ($\text{HR}$) \cite{islam2024comprehensive, cossio2025comprehensive}, defined as the proportion of generated responses containing at least one factually incorrect claim when benchmarked against the held-out, human-annotated Factual Test Set ($\mathcal{T}_{\text{Fact}}$). $\mathcal{T}_{\text{Fact}}$ consisted of evaluation prompts, sampled equally from the \textit{FEVER dataset} \cite{thorne2018fever} and the \textit{TruthfulQA benchmark} \cite{lin2021truthfulqa}. Expert annotators rated each model's output, achieving strong inter-annotator agreement.

\subsection{Phase 2: Architectural Intervention and Evaluation}
\label{sec:phase2_intervention}

\subsubsection{Integration of the Layer-Specific Factual Gate (LSFG)}
\label{sec:lsfg_integration}
To mitigate the persistent issue of factual drift in generative transformers \cite{islam2024comprehensive, cossio2025comprehensive}, we introduced the Layer-Specific Factual Gate ($\text{LSFG}$) into the decoder of the baseline model. Specifically, the intervention targets the final decoder layers, replacing their standard Feed-Forward Networks ($\text{FFN}$) \cite{vaswani2017attention} with a novel Gated Factual Network ($\text{GFN}$) The gating mechanism enforces selective suppression of activations contributing to factual inconsistency, formalized as:

\[
\text{Output} = g \odot \text{FFN}(z), \quad \text{where} \quad g = \sigma(W_g z + b_g)
\]

\noindent
where $z$ is the input to the FFN, $W_g$ and $b_g$ are the learned gating parameters, $\sigma$ is the element-wise sigmoid activation, and $\odot$ denotes Hadamard product. This gating mechanism provides a dynamic factual fidelity filter over the representational space of the terminal layers.

\subsubsection{Training Regime for the Experimental Model ($M_{\text{LSFG}}$)}
\label{sec:experimental_training}
The experimental model, designated $M_{\text{LSFG}}$, was trained exclusively on the High-Reliability Corpus ($\mathcal{D}_{\text{HR}}$) under an identical regimen to Baseline $B_2$ (training steps, batch size, optimizer configuration, and schedule). This ensured that any measured improvements could be unambiguously attributed to the $\text{LSFG}$ intervention rather than differences in training exposure.

The optimization objective was the \textbf{Standard Cross-Entropy Loss} with label smoothing \cite{vaswani2017attention}. This loss function implicitly optimized $W_g$ and $b_g$ to minimize the likelihood of factually incorrect generations during training.


\section{Results}
\begin{table}[ht]
\centering
\begin{tabular}{lrr}
\toprule
A & B & C \\
\midrule
A1 & B1 & C1\\
A2 & B2 & C2\\
A3 & B3 & C3\\
A4 & B4 & C4\\
\bottomrule
\end{tabular}
\caption{\label{tab:train}The summary of the table}
\end{table}

Reference to the Table \ref{tab:res} on page \pageref{tab:res} and a cite \cite{reversing}.

% ---------------------------------------------------------------


\section{Discussion} % Conclusions, Further research
Your discussion

\section*{Acknowledgment}
Your acknowledgement

\printbibliography[title=References]


\clearpage
\vspace{2cm}

\noindent\textbf{Zoltán Blahovics}\\
Department of Computer Science\\Eötvös Loránd University\\Pázmány Péter Sétány 1/C Budapest, Hungary\\
Budapest\\
Hungary\\
{\tt euxhhx@inf.elte.hu}\\

\noindent\textbf{Bence Nagy}\\
Department of Computer Science\\Eötvös Loránd University\\Pázmány Péter Sétány 1/C Budapest, Hungary\\
Budapest\\
Hungary\\
{\tt hvtdd4@inf.elte.hu}\\

\noindent\textbf{Krisztián Nemes-Kovács}\\
Department of Computer Science\\Eötvös Loránd University\\Pázmány Péter Sétány 1/C Budapest, Hungary\\
Budapest\\
Hungary\\
{\tt email}\\

\noindent\textbf{Balázs Fekete}\\
Department of Computer Science\\Eötvös Loránd University\\Pázmány Péter Sétány 1/C Budapest, Hungary\\
Budapest\\
Hungary\\
{\tt email}\\

\end{document}
